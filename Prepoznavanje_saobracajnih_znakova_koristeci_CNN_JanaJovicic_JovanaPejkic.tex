% !TEX encoding = UTF-8 Unicode
\documentclass[a4paper]{article}

% Dodato da se ne bi prikazivali podnaslovi u sadrzaju, a da ipak ostanu numerisani u okviru teksta
\setcounter{tocdepth}{1}    % levels under \subsection are not listed in the TOC


\usepackage{color}
\usepackage{url}
\usepackage[utf8]{inputenc} % make weird characters work
\usepackage{graphicx}


\usepackage[english,serbian]{babel}

\usepackage[unicode]{hyperref}
\hypersetup{colorlinks,citecolor=green,filecolor=green,linkcolor=blue,urlcolor=blue}

\usepackage{listings}

\newtheorem{primer}{Primer}[section]

\definecolor{mygreen}{rgb}{0,0.6,0}
\definecolor{mygray}{rgb}{0.7,0.7,0.7}
\definecolor{mymauve}{rgb}{0.58,0,0.82}

\lstset{ 
  backgroundcolor=\color{white},   % choose the background color; you must add \usepackage{color} or \usepackage{xcolor}; should come as last argument
  basicstyle=\scriptsize\ttfamily,        % the size of the fonts that are used for the code
  breakatwhitespace=false,         % sets if automatic breaks should only happen at whitespace
  breaklines=false,                 % sets automatic line breaking
  captionpos=b,                    % sets the caption-position to bottom
  commentstyle=\color{mygreen},    % comment style
  deletekeywords={...},            % if you want to delete keywords from the given language
  escapeinside={\%*}{*)},          % if you want to add LaTeX within your code
  extendedchars=true,              % lets you use non-ASCII characters; for 8-bits encodings only, does not work with UTF-8
%  firstnumber=1000,                % start line enumeration with line 1000
  frame=single,	                   % adds a frame around the code
  keepspaces=true,                 % keeps spaces in text, useful for keeping indentation of code (possibly needs columns=flexible)
  keywordstyle=\color{blue},       % keyword style
  language=SQL,                 % the language of the code
  morekeywords={*,...},            % if you want to add more keywords to the set
%  numbers=left,                    % where to put the line-numbers; possible values are (none, left, right)
%  numbersep=0pt,                   % how far the line-numbers are from the code
%  numberstyle=\tiny\color{mygray}, % the style that is used for the line-numbers
  rulecolor=\color{black},         % if not set, the frame-color may be changed on line-breaks within not-black text (e.g. comments (green here))
  showspaces=false,                % show spaces everywhere adding particular underscores; it overrides 'showstringspaces'
  showstringspaces=false,          % underline spaces within strings only
  showtabs=false,                  % show tabs within strings adding particular underscores
%  stepnumber=2,                    % the step between two line-numbers. If it's 1, each line will be numbered
  stringstyle=\color{mymauve},     % string literal style
  tabsize=2,	                   % sets default tabsize to 2 spaces
  title=\lstname                   % show the filename of files included with \lstinputlisting; also try caption instead of title
}

\renewcommand{\lstlistingname}{Kod}% Listing -> Algorithm
\renewcommand{\lstlistlistingname}{List of \lstlistingname s}%

\begin{document}

\title{Prepoznavanje saobraćajnih znakova koristeći CNN\\ \small{Seminarski rad u okviru kursa\\Računarska inteligencija\\ Matematički fakultet}}

\author{Jana Jovičić ???/2015, Jovana Pejkić 435/2016 \\ jana.jovicic755@gmail.com, jov4ana@gmail.com}

\date{16.~maj 2019.}

\maketitle

\abstract{

}

\tableofcontents

\newpage

\section{Uvod}
\label{sec:uvod}



\section{CNN}
\label{sec:cnn}



\subsection{Podnaslov 1}



\subsection{Podnaslov 2}

Tekst tekst tekst tekst tekst.

%\begin{figure}[h!]
%\begin{center}
%\includegraphics[scale=0.47]{naslov_slike.png}
%\end{center}
%\caption{Naslov slike}
%\label{fig:naslov_slike}
%\end{figure}


\section{Podnaslov 3}	
\label{sec:podnaslov3}

Tekst tekst tekst tekst tekst %\cite{ime_knjige}.


\section{Podnaslov 4}
\label{sec:podnaslov4}

Tekst tekst tekst tekst tekst %\ref{sec:ime_necega}.


\section{Podnaslov 5}
\label{sec:podnaslov5}

%\footnote{tekst koji se prikazuje kao fusnota}
%\textit{iskosen tekst}
%\ref{sec:primer_koda}


\begin{table}[h!]
\begin{center}
\caption{Primer tabele}
\begin{tabular}{|l|l|l|}
\hline
tekst & tekst & tekst tekst \\
\hline
tekst &  "tekst" &  tekst \\
\hline 
\end{tabular}
\label{tabela}
\end{center}
\end{table}

\subsection{Podnaslov 6}
\label{sec:podnaslov6}

Kod \ref{tabela1} demonstrira...

\begin{lstlisting}[caption={Primer koda},frame=single, label=tabela1]
table[key] = value
\end{lstlisting}


\subsection*{Podsekcija 1}


\subsubsection*{PodPodsekcija 1}


\subsection{Podnaslov 7}
\label{sec:podnaslov7}


\subsection{Podnaslov 8}
\label{sec:podnaslov8}


\section{Zaključak}
\label{sec:zakljucak}

\addcontentsline{toc}{section}{Literatura}
\appendix
\bibliography{seminarski} 
\bibliographystyle{plain}

\appendix
\section{Dodatak}
\subsection{Podnaslov 1}
%\label{sec:???}




\end{document}
